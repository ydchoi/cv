\documentclass{article} % For LaTeX2e
\usepackage{nips14submit_e,times}
\usepackage{amsmath}
\usepackage{amsthm}
\usepackage{amssymb}
\usepackage{mathtools}
\usepackage{hyperref}
\usepackage{url}
\usepackage{algorithm}
\usepackage[noend]{algpseudocode}
%\documentstyle[nips14submit_09,times,art10]{article} % For LaTeX 2.09

\usepackage{graphicx}
\usepackage{caption}
\usepackage{subcaption}

\def\eQb#1\eQe{\begin{eqnarray*}#1\end{eqnarray*}}

\providecommand{\e}[1]{\ensuremath{\times 10^{#1}}}
\providecommand{\pb}[0]{\pagebreak}


\newenvironment{claim}[1]{\par\noindent\underline{Claim:}\space#1}{}
\newtheoremstyle{quest}{\topsep}{\topsep}{}{}{\bfseries}{}{ }{\thmname{#1}\thmnote{ #3}.}
\theoremstyle{quest}
\newtheorem*{definition}{Definition}
\newtheorem*{theorem}{Theorem}
\newtheorem*{question}{Question}
\newtheorem*{exercise}{Exercise}
\newtheorem*{challengeproblem}{Challenge Problem}
\newtheorem*{solution}{Solution}
\usepackage{verbatimbox}
\usepackage{listings}
\title{Numerical Computing: Assignment IX}


\author{
Youngduck Choi \\
CILVR Lab \\
CIMS, New York University \\
\texttt{yc1104@nyu.edu} \\
}


% The \author macro works with any number of authors. There are two commands
% used to separate the names and addresses of multiple authors: \And and \AND.
%
% Using \And between authors leaves it to \LaTeX{} to determine where to break
% the lines. Using \AND forces a linebreak at that point. So, if \LaTeX{}
% puts 3 of 4 authors names on the first line, and the last on the second
% line, try using \AND instead of \And before the third author name.

\newcommand{\fix}{\marginpar{FIX}}
\newcommand{\new}{\marginpar{NEW}}

\nipsfinalcopy % Uncomment for camera-ready version

\begin{document}


\maketitle

\begin{abstract}
This ends my year long journey with Professor Wright.
\end{abstract}

\section{Solutions to the problems}

\begin{question}[9.1. Rank-one update]
\end{question}
\begin{solution}
\textbf{(a)}
Let $P$ be a matrix such that 
\[
P = I - \dfrac{ss^T}{(s^{T}s)}
\]
where $I$ is the $n$-dimensional identity and $s$ is a real vector.
We want to show that $P^2 = P$. Squaring the matrix $P$ yields
\eQb
P^2 &=& (I - \dfrac{ss^T}{(s^Ts)})^2. 
\eQe
Distributing the terms on the RHS we obtain
\eQb
P^2 &=& I\cdot I - I \cdot \dfrac{ss^T}{(s^{T}s)} - \dfrac{ss^T}{(s^{T}s)} \cdot I +  
\dfrac{ss^T}{(s^{T}s)} \cdot \dfrac{ss^T}{(s^{T}s)} 
\eQe
where $\cdot$ denotes the matrix multiplication with vector multiplication pre-defined. 
We know that for any arbitrary matrix $X$,
$I \cdot X = X \cdot I = X$ holds. Hence, we can simplify the above equation as
\eQb
P^2 &=& I - \dfrac{ss^T}{(s^{T}s)} - \dfrac{ss^T}{(s^{T}s)} + \dfrac{ss^T}{(s^{T}s)}
\cdot \dfrac{ss^T}{(s^{T}s)}.
\eQe
Notice that the third additive term on the right handside,
using the associativity of vector multiplication, can be viewed as
$\dfrac{s(s^{T}s)s^T}{(s^{T}s)^2}$. With the cancellations of the 
scalar terms $(s^Ts)$, we see that the third term equals $\dfrac{ss^T}{(s^Ts)}$.
Therefore,
one of the non-identity term cancels out with the third term and we have
\eQb
P^2 &=& I - \dfrac{ss^T}{(s^Ts)}, \\
\eQe
thereby showing that $P^2 = P$ as desired.

\pagebreak

\textbf{(b)}
We wish to compute the $n$ eigenvalues of $P$. Notice that $I - \dfrac{ss^T}{(s^Ts)}$ is a rank-one update
with a unit matrix norm, applied to the identity. Therefore, we have an eigenvalue of $1$ with $n-1$ multiplicity
and $0$ with multiplicity $1$. The eigenvector can be computed by reading of the rank position of the update matrix.
\end{solution}


\begin{question}[9.2 Eigenvalues of $AB$]
\end{question}
\begin{solution}
Let $A$ and $B$ be real $n \times n$ matrices, such that
\eQb
A = \begin{pmatrix}
1 & -1 \\
1 & 1 \\
\end{pmatrix} \text{ and  } 
B = \begin{pmatrix}
1 & 1 \\
0 & 1 \\
\end{pmatrix}.
\eQe
Denote the sets of eigenvalues of $A$ and $B$ as $\lambda_A$ and $\lambda_B$ respectively.
Notice that $A$ is chosen to be a rotational matrix of $\dfrac{\pi}{2}$ with $\sqrt{2}$ scaling,
which will yield complex eigenvalues $1 \pm i$. $B$ is a triangular matrix, and therefore we can
obtain its eigenvalues from the diagonal entries.
Then, we have that 
\eQb
\lambda_A &=& \{ 1 + i , 1 - i \} \\
\lambda_B &=& \{ 1 \},
\eQe
where the eigenvalue $1$ from $\lambda_B$ has a multiplicity of $2$. Consider the product
matrix $AB$, which can be written as 
\eQb
AB &=& \begin{pmatrix}
1 & 0 \\
1 & 2 \\
\end{pmatrix}
\eQe
As $AB$ is a triangular matrix the eigenvalue set of $AB$ is 
\[
\lambda_{AB} = \{ 1, 2 \}. 
\]
It is clear that $\lambda_{AB}$ cannot be constructed from the pairwise product form $\lambda_A$ and
$\lambda_B$. Hence, we have shown through a counter-example that eigenvalues of $AB$ are not
the pairwise products of the eigenvalues of $A$ and $B$.

\end{solution}


\pagebreak

\begin{question}[9.3 Inverse Power Method with Shift]
\end{question}
\begin{solution}
The implementation for inverse iteration with shift used in this exercise is included
in the appendix.

\smallskip

\textbf{(a)}
We perform $7$ steps of inverse iteration with 
\eQb
A = \begin{pmatrix}
3 & 10 \\
3 & 2 \\
\end{pmatrix}
\text{, }
y_0 = \begin{pmatrix}
1 \\
1 \\
\end{pmatrix}
\text{, and }
\sigma = 7.
\eQe

\begin{small}
\begin{verbbox}
EDU>> inverse_iteration
sigma =
     7
A =
     3    10
     3     2
I =
     1     0
     0     1
y_k =
     1
     1
A_shift =
    -4    10
     3    -5
k =

     2
lambda_k =

     7.454545454545454e+00
y_k =
     9.061831399952656e-01
     4.228854653311239e-01
err_k =
     5.454545454545459e-01
k =
     3
lambda_k =
     8.020104244229337e+00
y_k =
     8.931918083968501e-01
     4.496758759515176e-01
err_k =
     2.010424422933710e-02
k =
     4
lambda_k =
     7.998083667239799e+00
y_k =
     8.945501194069496e-01
     4.469676541641597e-01
err_k =
     1.916332760201200e-03
\end{verbbox}
\begin{figure}[h!]
\centering
\theverbbox
\caption{Inverse power method for $(a)$ upto k = 4}
\end{figure}
\end{small}

\pagebreak

\begin{small}
\begin{verbbox}
EDU>> inverse_iteration
k =
     5
lambda_k =
     8.000192586544665e+00
y_k =
     8.944148920726019e-01
     4.472381925090433e-01
err_k =
     1.925865446654740e-04
k =
     6
lambda_k =
     7.999980750865574e+00
y_k =
     8.944284208317694e-01
     4.472111358277962e-01
err_k =
     1.924913442596932e-05
k =
     7
lambda_k =
     8.000001925008656e+00
y_k =
     8.944270680161217e-01
     4.472138414674616e-01
err_k =
     1.925008655589977e-06
k =
     8
lambda_k =
     7.999999807500087e+00
y_k =
     8.944272032982891e-01
     4.472135709032105e-01
err_k =
     1.924999128988247e-07
\end{verbbox}
\begin{figure}[h!]
\centering
\theverbbox
\caption{Inverse power method for $(a)$ from k = 5 to k = 8}
\end{figure}
\end{small}
As the ratio of the largest-magnitude to next-largest magnitude eigenvalues is $\dfrac{1}{10}$,
we see rapid convergence to the true value of $\lambda = 8$.

\pagebreak

\textbf{(b)}
We perform $7$ steps of inverse iteration with 
\eQb
A = \begin{pmatrix}
3 & -1 \\
0 & 2 \\
\end{pmatrix}
\text{, }
y_0 = \begin{pmatrix}
-1 \\
1 \\
\end{pmatrix}
\text{, and }
\sigma = 3.1.
\eQe

\begin{small}
\begin{verbbox}
EDU>> inverse_iteration
sigma =
     3.100000000000000e+00
A =
     3    -1
     0     2
I =
     1     0
     0     1
y_k =
    -1
     1
A_shift =
    -1.000000000000001e-01    -1.000000000000000e+00
                         0    -1.100000000000000e+00
k =
     2
lambda_k =
     3.050000000000000e+00
y_k =
     9.988681377244375e-01
    -4.756514941544944e-02
err_k =
     5.000000000000027e-02
k =
     3
lambda_k =
     3.003951007506914e+00
y_k =
    -9.999913914439560e-01
     4.149341873211441e-03
err_k =
     3.951007506914461e-03
k =
     4
lambda_k =
     3.000374238658229e+00
y_k =
     9.999999293877241e-01
    -3.757985454294350e-04
err_k =
     3.742386582294976e-04
\end{verbbox}
\begin{figure}[h!]
\centering
\theverbbox
\caption{Inverse power method for $(b)$ upto k = 4}
\end{figure}
\end{small}

\pagebreak

\begin{small}
\begin{verbbox}
k =
     5
lambda_k =
     3.000034139004426e+00
y_k =
    -9.999999994168260e-01
     3.415183905661792e-05
err_k =
     3.413900442605922e-05
k =
     6
lambda_k =
     3.000003104510225e+00
y_k =
     9.999999999951807e-01
    -3.104616253892986e-06
err_k =
     3.104510225337975e-06
k =
     7
lambda_k =
     3.000000282236169e+00
y_k =
    -9.999999999999603e-01
     2.822370446846058e-07
err_k =
     2.822361686583008e-07
k =
     8
lambda_k =
     3.000000025657899e+00
y_k =
     9.999999999999998e-01
    -2.565790656986363e-08
err_k =
     2.565789936070928e-08
\end{verbbox}
\begin{figure}[h!]
\centering
\theverbbox
\caption{Inverse power method for $(b)$ from k = 5 to k = 8}
\end{figure}
\end{small}

Again, we observe fast convergence to the true largest eigenvalue, $\lambda = 3$.
Through computations, one can verify that $A_{\sigma}^{-1}$ yields a large value for 
the ratio of the largest-magnitude to next-largest magnitude eigenvalues, approximating
$\dfrac{1}{10}$ as before. 

\pagebreak

\textbf{(c)}
We perform $7$ steps of inverse iteration with 
\eQb
A = \begin{pmatrix}
3 & -1 \\
0 & 2 \\
\end{pmatrix}
\text{, }
y_0 = \begin{pmatrix}
1 \\
1 \\
\end{pmatrix}
\text{, and }
\sigma = 3.1.
\eQe

\begin{small}
\begin{verbbox}
K>> inverse_iteration
sigma =
     3.100000000000000e+00
A =
     3    -1
     0     2
I =
     1     0
     0     1
y_k =
     1
     1
A_shift =
    -1.000000000000001e-01    -1.000000000000000e+00
                         0    -1.100000000000000e+00
k =
     2
lambda_k =
     2.550000000000000e+00
y_k =
    -7.071067811865474e-01
    -7.071067811865477e-01
err_k =
     4.500000000000002e-01
k =
     3
lambda_k =
     1.999999999999998e+00
y_k =
     7.071067811865460e-01
     7.071067811865490e-01
err_k =
     1.000000000000002e+00
k =
     4
lambda_k =
     1.999999999999977e+00
y_k =
    -7.071067811865313e-01
    -7.071067811865638e-01
err_k =
     1.000000000000023e+00
\end{verbbox}
\begin{figure}[h!]
\centering
\theverbbox
\caption{Inverse power method for $(c)$ upto k = 4}
\end{figure}
\end{small}

\pagebreak

\begin{small}
\begin{verbbox}
k =
     5
lambda_k =
     1.999999999999747e+00
y_k =
     7.071067811863688e-01
     7.071067811867262e-01
err_k =
     1.000000000000253e+00
k =
     6
lambda_k =
     1.999999999997220e+00
y_k =
    -7.071067811845821e-01
    -7.071067811885129e-01
err_k =
     1.000000000002780e+00
k =
     7
lambda_k =
     1.999999999969426e+00
y_k =
     7.071067811649285e-01
     7.071067812081664e-01
err_k =
     1.000000000030574e+00
k =
     8
lambda_k =
     1.999999999663688e+00
y_k =
    -7.071067809487395e-01
    -7.071067814243556e-01
err_k =
     1.000000000336312e+00
\end{verbbox}
\begin{figure}[h!]
\centering
\theverbbox
\caption{Inverse power method for $(c)$ from k = 5 to k = 8}
\end{figure}
\end{small}
In this example, we clearly see that a poor choice of the initial value of $y$ leads
to divergence in the computation of the largest eigenvalue. The chosen $y = (1,-1)$ in this example
is not a good approximation to the target eigenvector as the previous example with $y = (1,1)$. 
Unlike the previous example,
the approximated eigenvalue, $\lambda_k$ does not converge to the true eigenvalue $\lambda = 3$.
\end{solution}

\pagebreak

\begin{question}[9.4 Rayleigh quotient iteration]
\end{question}
\begin{solution}
The implementation of Rayleigh quotient iteration used in this exercise is included in the 
appendix.

\smallskip

\textbf{(a)}
We compute the eigenvalues and eigenvectors of $A$. The below figure contains the
eigenvalues as the diagnoal entires of the $D$ matrix and the eigenvectors as the
columns of the matrix $V$.
\begin{small}
\begin{verbbox}
K>> A = [3 1 4; 1 3 4 ; 4 4 0]
A =
     3     1     4
     1     3     4
     4     4     0
K>> [V,D] = eig(A)
V =
4.082482904638630e-01     7.071067811865475e-01     5.773502691896256e-01
4.082482904638629e-01    -7.071067811865475e-01     5.773502691896255e-01
-8.164965809277259e-01                         0     5.773502691896258e-01
D =
-4.000000000000001e+00                         0                         0
                     0     2.000000000000000e+00                         0
                     0                         0     7.999999999999999e+00
\end{verbbox}
\begin{figure}[h!]
\centering
\theverbbox
\caption{Eigenvalues and eigenvectors of $A$}
\end{figure}
\end{small}

\smallskip

\textbf{(b)}
We take the eigenvectors from the first and second columns of $V$. Notice that they are
indeed unit vectors. Hence, by multiplying those two eigenvectors with $0.65$ scalar
we obtain the vectors that differ in norm by $0.35$.
\begin{small}
\begin{verbbox}
K>> y_1
y_1 =
     4.082482904638630e-01
     4.082482904638629e-01
    -8.164965809277259e-01
K>> y_1_adjusted
y_1_adjusted =
     2.653613888015109e-01
     2.653613888015109e-01
    -5.307227776030219e-01
K>> y_2
y_2 =
     7.071067811865475e-01
    -7.071067811865475e-01
                         0
K>> y_2_adjusted
y_2_adjusted =

     4.596194077712559e-01
    -4.596194077712559e-01
                         0
\end{verbbox}
\begin{figure}[h!]
\centering
\theverbbox
\caption{Adjusted Vectors}
\end{figure}
\end{small}

\pagebreak

\textbf{(c)}
We have two separate runs for using two adjusted vectors.
The first one uses the adjusted vector from the first eigenvector.
As one can see in the below figure, the iterates correctly converge to the 
eigenvalue in two iterations. The perturbed eigenvector converges to the
true eigenvector of the matrix $A$.
\begin{small}
\begin{verbbox}
K>> rayleigh
A =
     3     1     4
     1     3     4
     4     4     0
I =
     1     0     0
     0     1     0
     0     0     1
y_k =
     2.653613888015109e-01
     2.653613888015109e-01
    -5.307227776030219e-01
sigma_k =
    -1.690000000000000e+00
k =
     0
cond_k =
     4.194805194805193e+00
err_k =
     2.310000000000000e+00
z_k =
    -1.148750600872342e-01
    -1.148750600872341e-01
     2.297501201744683e-01
y_k =
    -4.082482904638631e-01
    -4.082482904638630e-01
     8.164965809277260e-01
sigma_k =
    -4
k =
     1
cond_k =
     6.938077456519682e+16
err_k =
     0
z_k =
    -2.757880273211542e+15
    -2.757880273211544e+15
     5.515760546423086e+15
y_k =
    -4.082482904638630e-01
    -4.082482904638631e-01
     8.164965809277260e-01
sigma_k =
    -4
\end{verbbox}
\begin{figure}[h!]
\centering
\theverbbox
\caption{Rayleigh with the adjusted vector from the first eigenvector}
\end{figure}
\end{small}

\pagebreak

The second run uses the adjusted vector from the second eigenvector. 
As one can see, the iterates do not converge and the condition number of
the matrix $A - \sigma_k I$ blows up even before the eigenvalue estimation
gets close to the true value. The perturbed eigenvector does not converge
to the real eigenvector from the matrix $A$.
\begin{small}
\begin{verbbox}
k =
     1
cond_k =
     2.903747280940554e+16
err_k =
     6
Warning: Matrix is close to singular or badly scaled.
Results may be inaccurate. RCOND =  4.194176e-17. 
> In rayleigh at 17
  In inverse_iteration at 16 
z_k =
    -1.685925442727410e+15
     1.685925442727410e+15
    -9.243225897863361e-03
y_k =
    -7.071067811865475e-01
     7.071067811865475e-01
    -3.876771502922898e-18
sigma_k =
     2.000000000000000e+00
k =
     2
cond_k =
     8.158537537569432e+15
err_k =
     6
Warning: Matrix is close to singular or badly scaled.
 Results may be inaccurate. RCOND =  4.194176e-17. 
> In rayleigh at 17
  In inverse_iteration at 16 
\end{verbbox}
\begin{figure}[h!]
\centering
\theverbbox
\caption{Rayleigh with the second adjusted vector}
\end{figure}
\end{small}

\end{solution}

\pagebreak

\begin{question}[9.5 Eigenvalues continued ]
\end{question}
\begin{solution}
Let $K$ be a symmetric $2 \times 2$ matrix, such that
\eQb
K &=& \begin{pmatrix}
h & a \\
a & 0 \\
\end{pmatrix},
\eQe
where $h$ and $a$ are real scalars.

\smallskip

\textbf{(a)} We wish to compute the eigenvalues of $K$. We know that 
eigenvalues of $K$, denoted as $\lambda$, can be obtained by solving
\[
det(A - \lambda I) = 0.
\]
Subtracting $\lambda$ from the diagonal terms from $K$, we obtain
\[
A - \lambda I = \begin{pmatrix}
h - \lambda & a \\
a & -\lambda \\
\end{pmatrix}.
\]
Using the determinant rule for $2 \times 2$ matrix, we get
\[
\lambda^2 - h\lambda - a^2 = 0.
\]
Applying the quadratic formula with respect to the $\lambda$ variable, we can obtain
an explicit expression for $\lambda$ in terms of $h$ and $a$ as follows:
\eQb
\lambda &=& \dfrac{h \pm \sqrt{h^2 + 4a^2}}{2}, \\
\eQe
thereby obtaining the expression for the two eigenvalues of $K$ explicitly.

\bigskip

\textbf{(b)} 
From the derivation in part $(a)$, we have that without loss of
generality, the two eigenvalues $\lambda_1$ and
$\lambda_2$ can be expressed as
\begin{eqnarray}
\lambda_1 &=& \dfrac{h + \sqrt{h^2 + 4a^2}}{2}, \\
\lambda_2 &=& \dfrac{h - \sqrt{h^2 + 4a^2}}{2}. 
\end{eqnarray}
$K$ will be nonsingular iff $h - a^2 \neq 0$, which directly follows
from the characterization of nonsingular matrices through determinants.
$K$ will have a single zero eigenvalue iff $h \neq 0$ and $a = 0$.
Assume that $h \neq 0$ and $a = 0$. From direct substitutions to
$\lambda_1$ and $\lambda_2$, we see that 
it is equivalent to $\lambda_1 \neq 0$ and
$\lambda_2 = 0$. $K$ cannot have two negative eigenvalues. We consider two separate
cases. First, assume that $h \geq 0$. Then, as $\sqrt{h^2 + 4a^2} \geq 0$, we have that
$\lambda_1 \geq 0$. Secondly, assume that $h < 0$. We have $\sqrt{h^2 + 4a^2} \geq -h$. Hence,
once again, $\lambda_1 \geq 0$ holds. Therefore, we have shown that $K$ cannot have two
eigenvalues.

\bigskip

\textbf{(c)} 
As before, without loss of
generality, we express $\lambda_1$ and $\lambda_2$ with $(1)$ and $(2)$ expressions respectively.
We consider to two possible cases. First, if $a = 0$, then we have $\lambda_2 = 0$.
Secondly, if $|a| > 0$, then we have $\sqrt{h^2 + 4a^2} > h$. We can further obtain that
$0 > h - \sqrt{h^2 + 4a^2}$ which implies that $\lambda_2$ is negative. Hence, we have shown that in all
cases the two eigenvalues of $K$ cannot both be positive.
\end{solution}

\pagebreak

\section{Appendix}
This sections contains the source codes written for the problem set.
\subsection{Inverse iteration with shift}
\lstinputlisting[language=Matlab]{inverse_iteration.m}
\subsection{Rayleigh quotient iteration}
\lstinputlisting[language=Matlab]{rayleigh.m}
\end{document}


